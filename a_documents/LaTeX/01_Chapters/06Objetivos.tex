\documentclass[../00_Main.tex]{subfiles}
\begin{document}

\subsection{General.}
Darle continuidad a los experimentos e investigaciones previas realizadas en el Instituto Tecnológico de Buenos Aires ITBA, analizando y obteniendo resultados del rendimiento en el algoritmo de machine learning que emula ondas ERP P300 usado en un experimento con interfaces cerebro computador en pacientes con ELA (Esclerosis Lateral Amiotrófica).

\subsection{Específicos.}

\subsubsection{} Realizar un recorrido en la mayor cantidad de información disponible sobre los objetos de estudio: por un lado, el potencial de eventos evocado P300. Por otro, electroencefalogramas de pacientes sanos y pacientes con ELA (Esclerosis Lateral Amiotrófica), y por último todo el conjunto de señales que se obtienen producto de una interfaz cerebro computador (BCI por sus siglas en inglés) vinculado a un instrumento de comunicación llamado P300 speller.

\subsubsection{} Completar un análisis exploratorio de datos (EDA por sus siglas en inglés) con los datasets usados en los experimentos, a través de las herramientas computacionales disponibles: librerías de python especializadas en el manejo de datos y electroencefalografía. 

%\subsubsection{\mbox{\textnormal{Completar un análisis exploratorio de datos}}}

%\subsubsection{\mbox{\fontsize{12}{Completar un análisis exploratorio de datos}}}

\hspace{2cm}\href{http://bnci-horizon-2020.eu/database/data-sets}{- BNCI Horizon 2020: 8. Speller P300 with ALS patients (008-2014)}.

\hspace{2cm}\href{https://www.kaggle.com/datasets/rramele/p300samplingdataset?resource=download}{- ITBA. P300 dataset of 8 healthy subjects}.

\subsubsection{} Modificar, en distintos rangos, las propiedades de la onda compuesta de manera sintética y verificar el rendimiento del algoritmo \href{https://github.com/faturita/python-nerv/blob/master/DrugSignal.py}{drugsignal.py}.


\subsubsection{} Presentar todos los resultados obtenidos de las modificaciones en las propiedades de la onda compuesta y proponer mejoras en la preconfiguración del algoritmo.

%\par\vspace*{\fill} % Moves keywords to the bottom of the page
%\textbf{\textit{Keywords --}} hardship, engineering, tired, verytired % Add you all the keywords associated with your thesis here

\biblio % Needed for referencing to working when compiling individual subfiles - Do not remove
\end{document}
