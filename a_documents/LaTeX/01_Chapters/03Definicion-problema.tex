\documentclass[../00_Main.tex]{subfiles}

\begin{document}

La continuidad de los proyectos de investigación es vital para obtener avances y mejoras en los resultados de los experimentos. Particularmente, las investigaciones en el tratamiento de señales electroencefalográficas con modelos de machine learning son un campo de estudio relativamente nuevo y con poca disponibilidad de datos, lo que genera obstáculos que impiden realizar experimentos comparativos a gran escala. En el caso de los experimentos descritos en el artículo \textit{EEG Waveform Analysis of P300 ERP with Applications to Brain Computer Interfaces} (\cite{EEGAnalysisBCI}) en pacientes con ELA (Esclerosis Lateral Amiotrófica), su continuidad se encontraba pausada por razones ajenas a este documento. Es posible que, una vez reanudada estas investigaciones, se puedan dar saltos posteriores con experimentos comparativos a gran escala.

Dentro del artículo descrito, se usó un algoritmo de machine learning que ensambla electroencefalogramas de pacientes pasivos; pacientes que participaron del experimento desconociendo las reglas de interacción con los equipos, con potenciales P300 en los lugares donde sabemos de antemano que suceden los eventos evocados. Es necesario analizar y testear dicho algoritmo para que la investigación continúe.
El EDA (Análisis Exploratorio de Datos, por sus siglas en inglés) y modificaciones en las propiedades de las ondas obtenidas de ese ensamble nos permitirán ampliar el abanico de resultados que arrojan distinta \textit{performance} en los resultados. Esto permitirá proponer mejoras en la preconfiguración del algoritmo.

\biblio % Needed for referencing to working when compiling individual subfiles - Do not remove

\end{document}