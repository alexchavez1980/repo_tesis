\documentclass[../00_Main.tex]{subfiles}

\begin{document}
La electroencefalografía es una de las herramientas clínicas que, a lo largo de las últimas décadas, se ha convertido en uno de los principales métodos para obtener imágenes en tiempo real del comportamiento cerebral de manera no invasiva, portátil y móvil más usado en el ambiente médico (\cite{Atlas}). Dentro de la electroencefalografía tenemos un conjunto de ondas con distintas características que varían en sus propiedades físicas como amplitud o frecuencia, como también en el origen y la ubicación en las distintas zonas del cerebro. La onda P300 se obtiene de ubicar un canal en el lóbulo parietal y su comportamiento es reactivo debido a estímulos esperados pero infrecuentes relacionados con actos cognitivos. 

Sin embargo, la electroencefalografía está expuesta a alteraciones no deseadas en sus resultados, ya que, por más controlado, preciso y consistente que sea el ambiente donde se realiza el experimento o la toma de muestra, estaremos sujetos a factores fuera de nuestro control. Esta problemática se suele afrontar generando ambientes de pruebas donde se pueda recrear la situación con la mayor fidelidad posible. Los experimentos pasados y éste trabajo integrador se cimentan en la base de datasets sintéticos, artificiales, con los que se simulan respuestas de ondas ERP P300 a partir de electroencefalogramas reales, con resultados previamente conocidos, a fin de trabajar en la performance de algoritmos que logren resultados con mejoras en el tiempo.

Los métodos y los procedimientos cuantitativos para automatizar la decodificación de ondas EEG como la P300 se basa en EEG no invasivo (\cite{IntroductionBCIR}). Sin embargo, los métodos de la decodificación de señales, basadas en detección de formas de onda, y además con algoritmos de machine learning son relativamente escasos. 

En este trabajo se dará continuidad a la investigación realizada en los experimentos descritos en el artículo \textit{EEG Waveform Analysis of P300 ERP with Applications to Brain Computer Interfaces}(\cite{EEGAnalysisBCI}) en pacientes con ELA (Esclerosis Lateral Amiotrófica): en éste trabajo se realiza un análisis exploratorio de electroencefalogramas llamados pasivos; pacientes que participaron del experimento pero desconociendo las reglas de interacción con el speller, explicado más adelante en profundidad. En una etapa posterior se “inyectan” potenciales P300 en los lugares donde sabemos de antemano suceden los eventos, y realizamos modificaciones en las propiedades de las ondas que arrojan distinta performance para permitirnos obtener mejoras en la preconfiguración del algoritmo.

% \subsection{Subsection}
% This is a subsection
% \subsection{Another subsection}

% Important stuff

% \begin{chronology}[5]{1}{5}{90ex}[\textwidth]
% \event{1}{Human Perception}
% \event{2}{Augmented Reality}
% \event{3}{Sensory Augmentation}
% \event{4}{Brain Computer-Interface Computational Imaging}
% \event{5}{Hybrid Team Collective Perception}
% \caption{Perception}
%\event{\decimaldate{25}{12}{2001}}{three}
% \label{fig:story}
% \end{chronology}

\biblio % Needed for referencing to working when compiling individual subfiles - Do not remove

\end{document}